\subsection{Motivation}

Industry developers have long spurned functional programming as a niche product that can only be used in either academics or a few rather exotic problem domains. The origin of this perception may have been that functional programming languages often "look" and work fundamentally different than more popular ones, e.g., from the C family.

That is not entirely dismissable. The first functional programming language, LISP, has been introduced 60 years ago by John McCarthy  (\cite{mccarthy_recursive_1959}). Based on the groundbreaking work of Alonzo Church on the Lambda calculus (\cite{church_properties_1936}; \cite{church_calculi_1941}), McCarthy created a whole new category of programming languages (\cite{turner_history_2012}). Those languages would become widespread and successful first and foremost in the scientific community in the decades to come.

However, since the early 2000s, functional programming has become more and more popular in the software industry as well. Programming languages that call themselves "functional-first," such as Scala and F\# have been introduced. And established "object-oriented" programming languages such as Java and C\# adopt functional ideas with accelerating speed.

Furthermore, it can regularly be observed that experienced developers with a background from the object-oriented world, who come into contact with functional programming, soon become almost enthusiastic: "Once you learn the benefits of functional programming, you find that it improves all the code you write. When I learned functional programming a few years ago, it re-energized my enthusiasm for programming. I saw new, exciting ways to approach old problems. The rigor of functional programming complemented the design and testing benefits of \textit{test-driven development}, giving me greater confidence in my work." (\cite[vii]{wampler_functional_2011})

At the same time, many companies still struggle to apply functional concepts and methods in concrete software projects. Although it can be assumed that doing so could provide them a competitive edge, as the example of Jet.com shows. For tactical reasons, Jet.com decided to adopt functional programming with the explicit goal of becoming an attractive target for the most talented developers on the market (\cite{han_how_2015}). The company was eventually sold to Walmart for \$3.3 billion only 2.5 years after it had been founded\footnote{https://techcrunch.com/2016/08/08/confirmed-walmart-buys-jet-com-for-3b-in-cash/ (retrieved August 8, 2019)}. 

So functional programming is an established and mature programming paradigm. A variety of programming languages support it. Many developers who try it quickly "get hooked." Also, from an economic point of view of a company, it is possible to build a conclusive argument for its introduction. But still, in practice, it does not always find application in all the fields it would presumably provide benefits over current mainstream approaches. 

That is the starting point of this thesis, which will explore the suitability of functional programming in one of the most popular areas of commercial software development of the last decade – mobile application development. It chooses the Xamarin platform which not only features object-oriented programming with the popular language C\# but also supports an often overlooked but powerful functional programming language: F\#. 

Hence this research question arises: How suited is functional programming with F\# compared to object-oriented programming with C\# for the development of mobile applications with Xamarin?

The author hypothesizes that the ecosystem around F\# has reached a state where it enables Xamarin developers to build comprehensive mobile applications. This is made possible by leveraging modern functional programming concepts, a rich set of libraries, and support by major IDEs. Functional programming, therefore, is a viable alternative to the established object-oriented approach with C\#. 

It is out of the question that mobile applications built on the Xamarin platform can at least technically be written in F\#. But how far does it go? This thesis will try to find out if it is possible to not only build such software with the "functional-first programming language" F\# but to leverage FP in a more holistic way. A way that not only benefits from a couple of functional features in the language itself. But a way that harnesses competitive advantages of FP by implementing concepts and utilizing libraries and frameworks deeply rooted in the F\# ecosystem. A way that produces results that can compete with those built through the traditional imperative and object-oriented approach. 

Secondary questions, which will also be covered: Why is FP still so uncommon in the software industry? Could C\# developers apply FP in their language as well? And what needs to be considered when C\# and F\# are being mixed together? 
